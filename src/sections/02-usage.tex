\section{Создание кластера и сервисов}\label{sec:usage}

\begin{frame}[fragile]{Инициализация кластера}

Docker Swarm необходимо предварительно инициализировать

\begin{tcolorbox-code}
\begin{lstlisting}[style=base]
$ docker @swarm init@

Swarm initialized: current node (r3ozii9s3hhwsv03p79gqzieb) is now a manager.

    To add a worker to this swarm, run the following command:

@docker swarm join --token SWMTKN-1-0q5ehdrl1649kfbpkma9ss5s1nlw1wvcmhuj1pcnnqnkquf9b9-8ftp1oriocxj9f50xi90wgj9z 192.168.0.103:2377@

To add a manager to this swarm, run 'docker swarm join-token manager' and follow the instructions.
\end{lstlisting}
\end{tcolorbox-code}

Удаление инстанса из кластера

\begin{tcolorbox-code}
\begin{lstlisting}[style=base]
$ docker @swarm leave@
\end{lstlisting}
\end{tcolorbox-code}
\end{frame}



\begin{frame}[fragile]{Добавление сервисов}

Сначала добавим сервис вручную

\begin{tcolorbox-code}
\begin{lstlisting}[style=base]
$ docker @service create@ --name=nginx nginx

w6w7uq1rfhug4kmhsrqh4jb3a
overall progress: 1 out of 1 tasks 
1/1: running   [==================================================>] 
verify: Service w6w7uq1rfhug4kmhsrqh4jb3a converged
    \end{lstlisting}
\end{tcolorbox-code}

\begin{tcolorbox-code}
\begin{lstlisting}[style=base]
$ docker service ls
ID             NAME            MODE         REPLICAS   IMAGE          PORTS
w6w7uq1rfhug   nginx           replicated   @1/1@        nginx:latest   
\end{lstlisting}
\end{tcolorbox-code}

Довольно утомительно выполнять для всех сервисов по-отдельности, не правда ли?

\end{frame}


\begin{frame}[fragile]{Docker Swarm и compose файлы}

Compose файлы используются \textit{в основном} только для первичной инициализации кластера

\begin{tcolorbox-code}
\begin{lstlisting}[style=base]
$ docker @stack deploy@ test -c docker-compose.yml
Creating network test_default
Creating service test_nginx

\end{lstlisting}
\end{tcolorbox-code}


\begin{tcolorbox-code}
\begin{lstlisting}[style=base]
services:
    nginx:
        image: nginx:latest
        @deploy@:
            replicas: 3
            update_config:
                parallelism: 1
                failure_action: rollback // pause, continue
                order: start-first // stop-first
                delay: 5s
            endpoint_mode: dnsrr // vip
\end{lstlisting}
\end{tcolorbox-code}

Swarm {\color{good} не будет обновлять сервис}, если \textbf{шаблон запуска} не изменился

\end{frame}

\begin{frame}[fragile]{Docker Swarm и compose файлы: одиночные задания}

Часто бывает необходимо запустить контейнер \textbf{один раз} при деплое или обновлении приложений в кластере


\begin{tcolorbox-code}
\begin{lstlisting}[style=base]
services:
    job:
        image: alpine
        deploy:
            @mode: global-job@
    \end{lstlisting}
\end{tcolorbox-code}

\begin{tcolorbox-code}
\begin{lstlisting}[style=base]
$ docker service ls
ID             NAME         MODE         REPLICAS              IMAGE           PORTS
qv535u46tsvc   test_job     @global job@   @0/0 (1/1 completed)@   alpine:latest   
\end{lstlisting}
\end{tcolorbox-code}

Подходит для выполнения миграций базы данных, создания индексов, очистки кэша и т.д.

\end{frame}

\section{Управление состоянием}\label{sec:management}
\begin{frame}[fragile]{Управление сервисами }

Основные команды для управление сервисами

\begin{tcolorbox-code}
\begin{lstlisting}[style=base]
$ @docker service@

Usage:  docker service COMMAND

Manage Swarm services

Commands:
  create      Create a new service
  @inspect@     Display detailed information on one or more services
  @logs@        Fetch the logs of a service or task
  @ls@          List services
  ps          List the tasks of one or more services
  rm          Remove one or more services
  rollback    Revert changes to a service configuration
  @scale@       Scale one or multiple replicated services
  @update@      Update a service
\end{lstlisting}
\end{tcolorbox-code}

\end{frame}

\begin{frame}[fragile]{Управление сервисами: \texttt{inspect} }

Принимает те же аргументы, что и \texttt{docker inspect}

Выводит информацию о сервисе в JSON формате

\begin{tcolorbox-code}
\begin{lstlisting}[style=base]
$ docker service @inspect@ nginx
[
    {
        "ID": "3uj6ue7i38369edtc9zq86bap",
        "Version": {
            "Index": 281
        },
        "CreatedAt": "2024-05-25T15:43:13.961333368Z",
        "UpdatedAt": "2024-05-25T15:43:50.922669814Z",
        "Spec": {
            "Name": "nginx",
...
\end{lstlisting}
\end{tcolorbox-code}

В выводе также доступны шаблон запуска и предыдущая конфигурация

\end{frame}

\begin{frame}[fragile]{Управление сервисами: \texttt{logs} }

Принимает те же аргументы, что и \texttt{docker logs}

\begin{tcolorbox-code}
\begin{lstlisting}[style=base]
$ docker service @logs@ nginx

@nginx@.@1@.@l1cmmvdtwgrr@:@home@    | /docker-entrypoint.sh: /docker-entrypoint.d/ is not empty, will attempt to perform configuration
\end{lstlisting}
\end{tcolorbox-code}

\begin{itemize}
    \item Название сервиса: \texttt{nginx}
    \item Номер реплики: \texttt{1}
    \item Идентификатор контейнера: \texttt{l1cmmvdtwgrr}
    \item Инстанс: \texttt{home}
\end{itemize}

\end{frame}

\begin{frame}[fragile]{Управление сервисами: \texttt{ls} }

\begin{tcolorbox-code}
\begin{lstlisting}[style=base]
$ docker service @ls@
ID             NAME      MODE         REPLICAS   IMAGE           PORTS
l2dmhictze8d   alpine    replicated   @1/1@        alpine:latest   
3uj6ue7i3836   nginx     replicated   @3/3@        nginx:latest
\end{lstlisting}
\end{tcolorbox-code}

\begin{itemize}
    \item Идентификатор сервиса: \texttt{3uj6ue7i3836}
    \item Название сервиса: \texttt{nginx}
    \item Тип сервиса: \texttt{replicated}
    \item Статус реплик: \texttt{3} (успешно запущенных) из \texttt{3} (запланированных)
    \item Образ: \texttt{nginx:latest}
\end{itemize}

\end{frame}

\begin{frame}[fragile]{Управление сервисами: \texttt{scale}}

Делается вручную или из compose файла

\begin{tcolorbox-code}
\begin{lstlisting}[style=base]
$ docker service @scale nginx=5@
nginx scaled to 5
overall progress: 5 out of 5 tasks 
1/5: running   [==================================================>] 
2/5: running   [==================================================>] 
3/5: running   [==================================================>] 
4/5: running   [==================================================>] 
5/5: running   [==================================================>] 
verify: Service nginx @converged@
\end{lstlisting}
\end{tcolorbox-code}

\begin{tcolorbox-code}
\begin{lstlisting}[style=base]
$ docker service ls
ID             NAME            MODE         REPLICAS   IMAGE          PORTS
w6w7uq1rfhug   nginx           replicated   @5/5@        nginx:latest   
\end{lstlisting}
\end{tcolorbox-code}

\end{frame}



\begin{frame}[fragile]{Управление сервисами: \texttt{update} \texttt{-{}-env-add/rm} }

Типичная задача --- обновление переменные окружения сервиса

\begin{tcolorbox-code}
\begin{lstlisting}[style=base]
$ docker service update @--env-add SOMEVAR=foobar@ nginx
1/1: running   [==================================================>] 
verify: Service nginx @converged@
\end{lstlisting}
\end{tcolorbox-code}

\begin{tcolorbox-code}
\begin{lstlisting}[style=base]
$ docker service inspect nginx
[{
    @"Spec"@: {
        "Name": "nginx",
        "TaskTemplate": {
            "ContainerSpec": {
                "Image": "nginx:latest@sha256:a484...",
                "Env": [
                    @"SOMEVAR=foobar"@
                ],
...
\end{lstlisting}
\end{tcolorbox-code}

\end{frame}

\begin{frame}[fragile]{Управление сервисами: \texttt{update} \texttt{-{}-image} }

Типичная задача --- обновление образа сервиса

\begin{tcolorbox-code}
\begin{lstlisting}[style=base]
$ docker service update @--image=nginx:1.25.5@ nginx
1/1: running   [==================================================>] 
verify: Service nginx @converged@
\end{lstlisting}
\end{tcolorbox-code}

\begin{tcolorbox-code}
\begin{lstlisting}[style=base]
$ docker service inspect nginx
[{      
    "Spec": {
        "TaskTemplate": {
            "ContainerSpec": {
                @"Image": "nginx:1.25.5@sha256:a484819eb60211f5299034ac80f6a681b06f89e65866ce91f356ed7c72af059c"@,
...
    \end{lstlisting}
\end{tcolorbox-code}

\begin{itemize}
    \item Jenkins: \texttt{env.BUILD\_NUMBER}
    \item Gitlab CI: \texttt{\$CI\_PIPELINE\_IID} и \texttt{\$CI\_COMMIT\_SHA}
\end{itemize}

Swarm {\color{bad} не будет обновлять сервис}, если образ не изменился

\end{frame}
